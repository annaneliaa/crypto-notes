\documentclass[10pt, oneside]{article} 
\usepackage{amsmath, amsthm, amssymb, calrsfs, wasysym, verbatim, bbm, color, graphics, graphicx, geometry, tocloft, subcaption, blindtext, hyperref, array}

\geometry{tmargin=.75in, bmargin=.75in, lmargin=.75in, rmargin = .75in}  

\newcommand{\R}{\mathbb{R}}
\newcommand{\C}{\mathbb{C}}
\newcommand{\Z}{\mathbb{Z}}
\newcommand{\N}{\mathbb{N}}
\newcommand{\Q}{\mathbb{Q}}
\newcommand{\Cdot}{\boldsymbol{\cdot}}

% comment command
\newcommand{\Comment}[1]{\noindent \textcolor{blue}{\textit{#1}} \par}

\newtheorem{thm}{Theorem}
\newtheorem{defn}{Definition}
\newtheorem{conv}{Convention}
\newtheorem{rem}{Remark}
\newtheorem{lem}{Lemma}
\newtheorem{cor}{Corollary}

\setlength{\parindent}{0pt}


\title{Security \& Cryptography Class Notes}
\author{Anna Visman}
\date{Academic Year 2024-2025}

\begin{document}

\maketitle
\tableofcontents

\vspace{.25in}

\newpage 
\section{Lecture 1}

\subsection{Security Overview}

A computer system is said to be secure if it satisfies the following properties:
\begin{itemize}
\item {\bf Confidentiality}: Unauthorized entities cannot access the system or its data
\item {\bf Integrity}: When you receive data, it is the right one
\item {\bf Availability}: The system or data is there when you need it
\end{itemize}

\begin{rem}
The mere presence of these properties does not necessarily mean that the system is fully secure in practice.
\end{rem}

A secure system is reliable:

\begin{itemize}
    \item Keep your personal data confidential 
    \item Allow only authorised access or modifications to resources
    \item Ensure that any produced results are correct
    \item Give you correct and meaningful results whenever you want them
\end{itemize}


Terminology:

\begin{itemize}
    \item {\bf{Assets}}: Things we want to protect (hardware, software, data)
    \item {\bf{Vulnerabilities}}: Weaknesses in a system that may be exploited in order to cause loss and harm
    \item {\bf{Threats}}: A loss or harm that might befall a system (interception, interruption, modification, fabrication)
    \item {\bf{Attack}}: An action which exploits a vulnerability to execute a threat 
    \item {\bf{Control/Defence}}: Removing/reducing a vulnerability. You control a vulnerability to prevent an attack and defend against a threat
\end{itemize}

Methods of Defence:

\begin{itemize}
    \item Prevent it
    \item Deter it: make the attack harder or more expensive
    \item Deflect it: make yourself less attractive to attacker
    \item Detect it: notice that the attack is occurring
    \item Recover from it: mitigate the effects of the attack
\end{itemize}

Principle of Easiest Penetration: A system is only as secure as its weakest link. An attacker will go after whatever part of the system is easiest for them, not most convenient for you. In order to build secure systems, we need to learn how to think like an attacker!

\subsection{Defense Overview}
Software controls:
\begin{itemize}
    \item Passwords and other forms of access control
    \item Operating systems separate users' actions from each other
    \item Virus scanner watch for malware
    \item Development controls enforce quality measures on the original source code 
    \item Personal firewalls that run on your desktop
\end{itemize}

Hardware controls:
\begin{itemize}
    \item Not usually protection of the hardware itself, but rather using separate hardware to protect the system as a whole 
    \item Fingerprint readers
    \item Smart tokens
    \item Firewalls
    \item Intrusion detection systems
\end{itemize}

Physical Systems:
\begin{itemize}
    \item Protection of the hardwell itself, as well as physical access to the console, storage media, etc.
    \item Locks
    \item Guards
    \item Off-site backups
\end{itemize}

Policies and Procedures:
\begin{itemize}
    \item Non-technical means can be used to protect against some classes of attack (e.g. VPNs for accessing interal company network)
    \item Rules about choosing Passwords
    \item Training in best security practices
\end{itemize}

\subsection{Cryptography Overview}
Objectives of Cryptography:
\begin{itemize}
    \item Protecting data privacy
    \item Authenticaion (message, data origin, entity)
    \item Non-repudiation: preventing the sender from later denying that they sent the message
\end{itemize}

\begin{defn}
Kerkhoff's Principle: The adversary knows all details about a crypto system except the secret key.
\end{defn}

\begin{defn}
Cipher: A method or algorithm used to transform readable data (called plaintext) into an unreadable format (called ciphertext) to protect its confidentiality.
\end{defn}

Encryption is the process of converting plaintext into ciphertext. Decryption is the reverse process. Encryption uses the key k, decryption uses the key k'. If k = k', the system is symmetric. If k $\neq$ k', the system is asymmetric. Decryption(Encryption(m)) = m.
\begin{figure}[h!]
    \centering
    \includegraphics[scale=0.5]{img/w1encryption.png}
    \caption{Encryption}
\end{figure}

\begin{table}[h!]
    \centering
    \begin{tabular}{|l|l|l|}
    \hline
    \textbf{Feature}        & \textbf{Private Key Encryption}        & \textbf{Public Key Encryption}         \\ \hline
    \textbf{Keys}           & Same key for encryption \& decryption  & Two keys: public and private           \\ \hline
    \textbf{Speed}          & Faster                                 & Slower                                 \\ \hline
    \textbf{Key sharing}    & Must be kept secret                    & Only the private key is secret         \\ \hline
    \textbf{Use cases}      & Encrypting large data, e.g., files     & Secure key exchange, digital signatures \\ \hline
    \end{tabular}
    \caption{Summary of Differences Between Private and Public Key Encryption}
    \label{tab:encryption_comparison}
    \end{table}
    
\subsection{Topics Covered in Course}
\begin{itemize}
\item Classical systems: simple ciphers, substitution, permutation, transposition, Caesar, Vigenere
\item Information Theoretic Security
\item Defining security: pseudorandomness, one-way functions, trapdoor functions
\item Notions of security: perfect secrecy, semantic security, IND security
\item Attacks on encryption schemes: objective, levels of computing power, amount of information available
\item Types attacks: ciphertext-only, known plaintext, chosen plaintext, chosen ciphertext, adaptive
\item Different types of adversaries: unbounded/polynomial computing power
\item Security: unconditionally secure, computationally secure
\item and more... see slides
\end{itemize}


\newpage

\section{Lecture 2}

\newcommand{\floor}[1]{\left\lfloor #1 \right\rfloor}

\subsection{Number Theory}

\subsubsection{Modular Arithmetic}
\begin{defn}
A positive integer \( N \) is called the \emph{modulus}. Two integers \( a \) and \( b \) are said to be congruent modulo \( N \), written \( a \equiv b \pmod{N} \), if \( N \) divides \( b - a \).
\end{defn}    

Examples:
\[
18 \equiv 4 \pmod{7}, \quad -18 \equiv 3 \pmod{7}.
\]

The set of integers modulo \( N \) is denoted by \( \mathbb{Z}/N\mathbb{Z} \) or \( \mathbb{Z}_N \):
\[
\mathbb{Z}/N\mathbb{Z} = \{ 0, 1, \dots, N-1 \}, \quad \#(\mathbb{Z}/N\mathbb{Z}) = N.
\]

Properties of Modular Arithmetic:
\begin{enumerate}
    \item Addition is closed: \(\forall a, b \in \mathbb{Z}/N\mathbb{Z} : a + b \in \mathbb{Z}/N\mathbb{Z}\).
    \item Addition is associative: \(\forall a, b, c \in \mathbb{Z}/N\mathbb{Z} : (a + b) + c = a + (b + c)\).
    \item \(0\) is an additive identity: \(\forall a \in \mathbb{Z}/N\mathbb{Z} : a + 0 = 0 + a = a\).
    \item The additive inverse always exists: \(\forall a \in \mathbb{Z}/N\mathbb{Z} : a + (N - a) = (N - a) + a = 0\).
    \item Addition is commutative: \(\forall a, b \in \mathbb{Z}/N\mathbb{Z} : a + b = b + a\).
    \item Multiplication is closed: \(\forall a, b \in \mathbb{Z}/N\mathbb{Z} : a \cdot b \in \mathbb{Z}/N\mathbb{Z}\).
    \item Multiplication is associative: \(\forall a, b, c \in \mathbb{Z}/N\mathbb{Z} : (a \cdot b) \cdot c = a \cdot (b \cdot c)\).
    \item \(1\) is a multiplicative identity: \(\forall a \in \mathbb{Z}/N\mathbb{Z} : a \cdot 1 = 1 \cdot a = a\).
    \item Multiplication and addition satisfy the distributive law: \(\forall a, b, c \in \mathbb{Z}/N\mathbb{Z} : (a + b) \cdot c = a \cdot c + b \cdot c\).
    \item Multiplication is commutative: \(\forall a, b \in \mathbb{Z}/N\mathbb{Z} : a \cdot b = b \cdot a\).
\end{enumerate}

\subsubsection{Modular Exponentiation}
Modular exponentiation is a technique used to efficiently compute expressions of the form \( a^b \mod m \), especially for large \( b \). The key idea is to repeatedly square the base \( a \), reduce modulo \( m \) at each step, and combine results as needed. \\

\textbf{Example: Compute \( 3^4 \mod 11 \)}

\begin{enumerate}
    \item Write the problem:
    \[
    3^4 \mod 11
    \]
 
 \item Break it into smaller steps using properties of modular arithmetic:
 \begin{enumerate}
    \item First, compute \( 3^2 \mod 11 \):
    \[
    3^2 = 9 \quad \Rightarrow \quad 9 \mod 11 = 9
    \]
    \item Then, square the result to get \( 3^4 \mod 11 \):
    \[
    3^4 = (3^2)^2 = 9^2 = 81 \quad \Rightarrow \quad 81 \mod 11 = 4
    \]
 \end{enumerate}
 
 \item Final result:
    \[
    3^4 \mod 11 = 4
    \]
 
\end{enumerate}
 
\textbf{General Algorithm: Exponentiation by Squaring}
\begin{enumerate}
    \item If \( b \) is even:
    \[
    a^b \mod m = \left( a^{b/2} \mod m \right)^2 \mod m
    \]
    \item If \( b \) is odd:
    \[
    a^b \mod m = \left( a \cdot a^{b-1} \mod m \right) \mod m
    \]
\end{enumerate}

You can also simplify the problem by reducing the base modulo:
\begin{defn}
    For any a, b, n, if $a \equiv b \pmod{n}$, then $a^k \equiv b^k \pmod{n}$ for any positive integer k.
\end{defn}
See an example of this in practice session 1 exercise 1e. 

\subsubsection{Groups and Rings}
\begin{defn}
    A \emph{group} is a set with an operation that is:
    \begin{itemize}
        \item Closed,
        \item Has an identity element,
        \item Associative, and
        \item Each element has an inverse.
    \end{itemize}
    \end{defn}
    
    \begin{defn}
    A group is \emph{abelian} if it is also commutative.
    \end{defn}
    
    Examples:
    \begin{itemize}
        \item The integers under addition (\( \mathbb{Z}, + \)), where the identity is \( 0 \) and the inverse of \( x \) is \( -x \).
        \item The nonzero rationals under multiplication (\( \mathbb{Q}^*, \cdot \)), where the identity is \( 1 \) and the inverse of \( x \) is \( 1/x \).
    \end{itemize}


Group types:
\begin{itemize}
    \item Multiplicative group: operation is multiplication.
    \item Additive group: operation is addition.
    \item Cyclic abelian group: generated by a single element.
\end{itemize}

\begin{defn}
An abelian group G is called \emph{cyclic} if there exists an element in the group, called the \emph{generator}, from which every other element in G can be obtained either by repeated application of the group operation to the generator, or by the use of the inverse operation.
\begin{itemize}
    \item If the group operation is multiplication ($(G, \cdot)$), a generator $g$ produces all elements by repeated multiplication or division: $h = g^x$, where $h$ is an arbitrary element in the group.  
    \item In modular arithmetic, $g$ is a generator if $g^x \mod m $ produces all nonzero elements of the group as $x$ varies.
\end{itemize}
\end{defn}

\textbf{Example:} The group \( \mathbb{Z}_7^* \) (the multiplicative group of integers modulo \( 7 \)) consists of the nonzero integers modulo \( 7 \) under multiplication. The elements of the group are:
\[
\mathbb{Z}_7^* = \{ 1, 2, 3, 4, 5, 6 \}.
\]

An element \( g \in \mathbb{Z}_7^* \) is a generator if the powers \( g^x \mod 7 \) (for \( x = 1, 2, 3, \dots, 6 \)) produce \textbf{all elements} of \( \mathbb{Z}_7^* \) exactly once. Let's test whether \( 3 \) is a generator:

\begin{enumerate}
    \item Compute the powers of \( 3 \) modulo \( 7 \):
    \[
    3^1 \mod 7 = 3,
    \]
    \[
    3^2 \mod 7 = 9 \mod 7 = 2,
    \]
    \[
    3^3 \mod 7 = 27 \mod 7 = 6,
    \]
    \[
    3^4 \mod 7 = 81 \mod 7 = 4,
    \]
    \[
    3^5 \mod 7 = 243 \mod 7 = 5,
    \]
    \[
    3^6 \mod 7 = 729 \mod 7 = 1.
    \]

    \item The results are:
    \[
    \{ 3, 2, 6, 4, 5, 1 \}.
    \]
\end{enumerate}

Since this list contains all elements of \( \mathbb{Z}_7^* \), \( 3 \) is a generator of \( \mathbb{Z}_7^* \). Other generators of \( \mathbb{Z}_7^* \) include \( 5 \). You can verify this by computing \( 5^x \mod 7 \) for \( x = 1, 2, \dots, 6 \).

\begin{defn}
    A \emph{ring} is a set with two operations (\(+\), \(\cdot\)) satisfying:
    \begin{itemize}
        \item The set is an abelian group under addition.
        \item Multiplication is associative and closed.
        \item Distributive laws hold.
    \end{itemize}
    \end{defn}

If multiplication is commutative, the ring is called \emph{commutative}. Examples:
\begin{itemize}
        \item Integers, real numbers, and complex numbers form infinite rings.
        \item \( \mathbb{Z}/N\mathbb{Z} \) forms a finite ring.
\end{itemize}
    
\subsubsection{Primes and Divisibility}
\begin{defn}
    An integer \( a \) divides another integer \( b \), denoted \( a \mid b \), if \( b = k \cdot a \) for some integer \( k \).
    \end{defn}
    
\begin{defn}
    A number \( p \) is \emph{prime} if its only divisors are \( 1 \) and \( p \).
    \end{defn}
    
Examples of primes: \( 2, 3, 5, 7, 11, \dots \).

\begin{defn}
Greatest Common Divisor: \( c = \gcd(a,b) \) if and only if \( c \) is the largest number that divides
both \( a \) and \( b \).
\end{defn}

\begin{thm}
Every positive integer can be written as a product of primes in a unique way.
\end{thm}

\begin{defn}
Two integers a and b are coprime, relatively prime or mutually prime if the only positive integer that is a divisor of both of them is 1.
\end{defn}

\begin{defn}
Euler's Totient Function: \( \phi(p) \) is the number of integers less than \( p \) that are relatively prime to \( p \).
\begin{itemize}
    \item If N is a prime then \( \phi(N) = N - 1 \).
    \item If p and q are both prime and \( p \neq q \), then \( \phi(pq) = (p-1)(q-1)\)
\end{itemize}

\[ \phi(N) = N \left(1 - \frac{1}{p_1}\right) \left(1 - \frac{1}{p_2}\right) \ldots \left(1 - \frac{1}{p_k}\right) = n \prod_{p | n} (1 - \frac{1}{p})\]

where \( p_1, p_2, \ldots, p_k \) are the prime factors of \( N \).
\end{defn}

Euler's Totient function counts the number of positive up to a given integer N that are relatively prime to N. \\

\subsubsection{Linear Congruences}
\textbf{Finding the solution to the linear congruence equation:} \[ a \cdot x \equiv b \pmod{N}\]

We want to know how many solutions exist for x modulo N given the coefficients a, b, and the modulus N.

\begin{enumerate}
    \item Compute the greatest common divisor (\(\gcd\)) of \(a\) and \(N\), denoted as \(\gcd(a, N) = g\).
    \item The following cases determine the number of solutions:
    \begin{enumerate}
        \item \textbf{If \(g = 1\):}
        \begin{itemize}
            \item When \(a\) and \(N\) are coprime (\(\gcd(a, N) = 1\)), the equation has \textbf{exactly one solution} modulo \(N\). This is because \(a\) has a multiplicative inverse modulo \(N\).
        \end{itemize}
        
        \item \textbf{If \(g > 1\) and \(g \mid b\):}
        \begin{itemize}
            \item If \(\gcd(a, N) = g > 1\) and \(g\) divides \(b\), then there are \textbf{exactly \(g\) solutions} modulo \(N\).
            \item These solutions can be determined by reducing the equation to a simpler congruence modulo \(N/g\).
        \end{itemize}
        
        \item \textbf{If \(g > 1\) and \(g \nmid b\):}
        \begin{itemize}
            \item If \(g\) does not divide \(b\), then the equation has \textbf{no solution}. This is because \(b\) is not in the span of \(a\) modulo \(N\).
        \end{itemize}
    \end{enumerate}
\end{enumerate}

\begin{defn}
    Multiplicative Inverse Modulo N: A number that, when multiplied by a given number a, gives a result of 1 modulo N. In other words, the multiplicative inverse of \(a \) modulo N is a number \(x \) such that:
    \[ a \cdot x \equiv 1 \pmod{N} \]

    \begin{itemize}
        \item The multiplicative inverse of \(a\) modulo \(N\) is denoted as \(a^{-1}\).
        \item A multiplicative inverse of \(a\) modulo \(N\) exists only if \(a\) and \(N\) are coprime, i.e., \(\gcd(a, N) = 1\).

        \item If \(a\) and \(N\) are not coprime, it’s impossible to find \(x\) such that
        \(
        a \cdot x \equiv 1 \pmod{N}.
        \)
        \item When N is a prime p, then for all non-zero values of \( a \in \mathbb{Z}/p\mathbb{Z} \) we always obtain a unique solution to the equation \( a \cdot x \equiv 1 \pmod{p} \).
        
    \end{itemize}
\end{defn}

Inverse in this case means that the two numbers multiply to 1 modulo N. Think about regular numbers: the inverse of 2 is \( \frac{1}{2}\) under multiplication, because \( 2* \frac{1}{2} = 1\). 

\subsubsection{Fields}
\begin{defn}
A field is a set G with two operations \((G, \cdot, +) \). It satisfies the following properties:
\begin{itemize}
    \item \((G, +)\) is an abelian group with identity element 0 (G is a commutative group under addition).
    \item \((G \backslash \{0\}, \cdot)\) is an abelian group (\(G \backslash \{0\}\) is a commutative group under multiplicatio).
    \item Multiplication distributes over addition, i.e., \( (G, \cdot, +) \) satisfies the distributive law.
    \end{itemize}
\end{defn}

A field is like the "ideal playground" for numbers: You can add, subtract, multiply, and divide (except by 0). Both addition and multiplication behave nicely (associative, commutative, etc.). Examples of fields include familiar systems like real numbers and rational numbers.
The key difference between rings and fields is that in a ring, division is not always possible. In a field, division (except by 0) is always possible, because every nonzero element has a multiplicative inverse.

\[ \mathbb{Z}/N\mathbb{Z} \] is a field if and only if N is prime (because then every nonzero element has a multiplicative inverse). Else, it is a ring. 

Think of \(\mathbb{Z}/N\mathbb{Z}\) as a "clock" with \(N\) hours. Once you pass \(N-1\), you wrap around back to \(0\). Arithmetic in \(\mathbb{Z}/N\mathbb{Z}\) always "cycles" within the set \(\{0, 1, \dots, N-1\}\). \\

\( (\mathbb{Z}/N\mathbb{Z})^* \) is the set of all elements that are invertible (the set of elements that are coprime to N). 

\[ (\mathbb{Z}/N\mathbb{Z})^* = \{ x \in \mathbb{Z}/N\mathbb{Z} : \gcd{(x, N)} = 1 \}\] 

The size of \( (\mathbb{Z}/N\mathbb{Z})^* \) is given by Euler's Totient function: \( \phi(N) \). If N is a prime p, then \( (\mathbb{Z}/N\mathbb{Z})^*  = \{1, ..., p-1 \}\).

\subsubsection{Lagrange's Theorem}
Lagrange's Theorem states that if \((G, \cdot)\) is a finite group with order (size) \(n = \#G\), then for any element \(a \in G\), the order of \(a\) (the smallest positive integer \(k\) such that \(a^k = 1\)) divides \(n\). In particular, it follows that:
\[
a^n = 1 \quad \text{for all } a \in G.
\]

\textbf{Application in Modular Arithmetic:}  
In the context of modular arithmetic, consider the group of units \(\mathbb{Z}/N\mathbb{Z}^*\) (the set of integers modulo \(N\) that are coprime to \(N\), with multiplication as the group operation). If \(x \in \mathbb{Z}/N\mathbb{Z}^*\), then the group has size \(\phi(N)\), where \(\phi(N)\) is Euler's totient function (the count of integers less than \(N\) that are coprime to \(N\)). Therefore:
\[
x^{\phi(N)} \equiv 1 \pmod{N}.
\]

\subsubsection{Fermat's Little Theorem}
Fermat's Little Theorem states that if \(p\) is a prime number and \(a\) is any integer, then:
\[
a^p \equiv a \pmod{p}.
\]

If \(a\) is not divisible by \(p\), then this can be rewritten as:
\[
a^{p-1} \equiv 1 \pmod{p}.
\]

\textbf{Explanation:}  
This theorem tells us that raising \(a\) to the power of \(p-1\) gives a remainder of \(1\) when divided by \(p\), provided \(a\) and \(p\) are coprime. Fermat's Little Theorem is useful for simplifying modular exponentiation and serves as a foundation for more advanced results like Euler's theorem.

\subsection{Basic Algorithms}

\subsubsection{Euclid's GCD Algorithm}

The \textbf{Greatest Common Divisor} (GCD) of two integers \(a\) and \(b\) is the largest integer \(d\) such that \(d\) divides both \(a\) and \(b\). \\

\textbf{Key Idea:} If we could factorize \(a\) and \(b\), we could easily determine their GCD. For example, consider:
\[
a = 2^4 \cdot 157 \cdot 4513^3, \quad b = 2^2 \cdot 157 \cdot 2269^3 \cdot 4513.
\]
Here, the GCD is given by:
\[
\text{gcd}(a, b) = 2^2 \cdot 157 \cdot 4513 = 2{,}834{,}164.
\]
However, computing prime factorizations is often impractical for large numbers. Instead, we use Euclid's Algorithm. \\

The Euclidean Algorithm is based on the principle:
\[
\text{gcd}(a, b) = \text{gcd}(a \mod b, b).
\]
The algorithm starts with two numbers \(a,b\), where \(a > b\). The remainder \(r = a \mod b\) is computed. Then, \(a\) is repeatedly replaced with \(b\) (the smaller number), and \(b\) with \(r\) (the remainder). This process continues until the remainder is \(0\). The last non-zero remainder (b) is the GCD of \(a\) and \(b\).

\textbf{Steps:}
\begin{enumerate}
    \item Let \(r_0 = a\) and \(r_1 = b\).
    \item Compute remainders \(r_2, r_3, \dots\) using:
    \[
    r_{i+2} = r_i \mod r_{i+1}, \quad \text{where } r_{i+2} < r_{i+1}.
    \]
    \item Stop when \(r_{m+1} = 0\). The GCD is \(r_m\).
\end{enumerate}

\textbf{Example:}
Compute \(\text{gcd}(21, 12)\):
\[
\begin{aligned}
\text{gcd}(21, 12) &= \text{gcd}(21 \mod 12, 12) = \text{gcd}(9, 12), \\
\text{gcd}(9, 12) &= \text{gcd}(12 \mod 9, 9) = \text{gcd}(3, 9), \\
\text{gcd}(3, 9) &= \text{gcd}(9 \mod 3, 3) = \text{gcd}(0, 3).
\end{aligned}
\]
Thus, \(\text{gcd}(21, 12) = 3\).

\subsubsection{The Extended Euclidean Algorithm}
In addition to computing the GCD, the Extended Euclidean Algorithm finds integers \(x\) and \(y\) such that:
\[
\text{gcd}(a, b) = ax + by = r.
\]
This is useful in many applications, such as finding modular inverses. For \(
\text{gcd}(a, b) = d
\) where \(d = 1 \), we can compute \(ax + yN = 1 \). Here \textbf{x} is the multiplicative inverse of \(a\) in modulo N. So, if \(\text{gcd}(a, N) = 1\), then \(a^{-1} \mod N = x \mod N\). \\

\textbf{Algorithm:}
\begin{enumerate}
    \item Start with \(r_0 = a\), \(r_1 = b\), \(s_0 = 1\), \(s_1 = 0\), \(t_0 = 0\), \(t_1 = 1\).
    \item For each step, compute:
    \[
    q_i = \left\lfloor \frac{r_{i-1}}{r_i} \right\rfloor, \quad r_{i+1} = r_{i-1} - q_i r_i,
    \]
    \[
    s_{i+1} = s_{i-1} - q_i s_i, \quad t_{i+1} = t_{i-1} - q_i t_i.
    \]
    \item Stop when \(r_{i+1} = 0\). Then, \(\text{gcd}(a, b) = r_i\), and \(x = s_i\), \(y = t_i\).
    
\end{enumerate}

\textbf{Example:}
Compute \(\text{gcd}(36, 24)\) and coefficients \(x, y\):
\[
\begin{aligned}
&\text{Step 1: } q = \left\lfloor \frac{36}{24} \right\rfloor = 1, \quad r = 36 - 1 \cdot 24 = 12, \\
&\text{Update: } x = 0 - 1 \cdot 1 = -1, \quad y = 1 - 1 \cdot 0 = 1, \\
&\text{Step 2: } q = \left\lfloor \frac{24}{12} \right\rfloor = 2, \quad r = 24 - 2 \cdot 12 = 0, \\
&\text{Update: } x = 1 - 2 \cdot (-1) = 3, \quad y = 0 - 2 \cdot 1 = -2.
\end{aligned}
\]
Thus, \(\text{gcd}(36, 24) = 12\), with \(x = -1\), \(y = 1\).

\subsubsection{The Chinese Remainder Theorem}
Let \(m_1, ..., m_r \) be pairwise relatively prime (i.e., \(\gcd(m_i, m_j) = 1\) for all \(i \neq j\)). Let \(x = a_i \mod m_i \) for all i. The CRT guarantees a unique solution given by:

\[ x = \sum^r_{i=1} a_iM_iy_i \mod M\] 

where

\[ M_i = M \slash \ m_i\]

and 

\[ y_i = M^{-1}_i \mod m_i \] 

$y_i$ is the modular inverse of $M_i$ modulo  $m_i$ (this can be computed using the Extended Euclidean Algorithm). The theorem is a way to solve a system of simultaneous modular congruences, finding a unique solution for a number that satisfies multiple modular equations, provided that the moduli are coprime/relatively prime. 

\[ x \equiv a_1 \mod m_1 \] 
\[ x \equiv a_2 \mod m_2 \]
\[...\]
\[ x \equiv a_k \mod m_k \]

In that case, there exists a \textbf{unique solution} \(x\) modulo \(M\), where \(M\) is the product of all the moduli:

\[
M = m_1 \cdot m_2 \cdot \ldots \cdot m_k.
\]

\textbf{Example}:
\[x \equiv 5 \mod 7 \]
\[x \equiv 3 \mod 11 \]
\[x \equiv 10 \mod 13 \]

Then $ M = 7 \cdot 11 \cdot 13 = 1001$. $M_1 = 1001 \slash 7 = 143$, $M_2 = 1001 \slash 11 = 91$, $M_3 = 1001 \slash 13 = 77$. $y_1 = 5, y_2 = 4, y_3 = 12$.
\[ x = \sum^r_{i=1} a_iM_iy_i \mod M = 5 \cdot 143 \cdot 5 + 3 \cdot 91 \cdot 4 + 10 \cdot 77 \cdot 12 \mod 1001 \] 
\[= 3575 + 1092 + 9240 \mod 1001 = 13907 \mod 1001 = 894\]

\Comment{add here why we need the CRT}

\subsubsection{Computing Legendre and Jacobi Symbols}

\begin{defn}
    Let \( n \) be a positive integer. An integer \( a \) is called a \textbf{quadratic residue modulo \( n \)} if there exists an integer \( x \) such that:
    \[
    x^2 \equiv a \pmod{n}.
    \]
    In other words, \( a \) is a quadratic residue modulo \( n \) if \( a \) is congruent to the square of some integer \( x \) modulo \( n \).
    
    If no such \( x \) exists, then \( a \) is called a \textbf{quadratic non-residue modulo \( n \)}.    
\end{defn}

\textbf{Example:} For \( n = 7 \), the integers modulo \( 7 \) are \( \{ 0, 1, 2, 3, 4, 5, 6 \} \). Computing the squares of each integer modulo \( 7 \):
\[
\begin{aligned}
    0^2 &\equiv 0 \pmod{7}, \\
    1^2 &\equiv 1 \pmod{7}, \\
    2^2 &\equiv 4 \pmod{7}, \\
    3^2 &\equiv 2 \pmod{7}, \\
    4^2 &\equiv 2 \pmod{7}, \\
    5^2 &\equiv 4 \pmod{7}, \\
    6^2 &\equiv 1 \pmod{7}.
\end{aligned}
\]

The quadratic residues modulo \( 7 \) are:
\[
\{ 0, 1, 2, 4 \}.
\]

The quadratic non-residues modulo \( 7 \) are:
\[
\{ 3, 5, 6 \}.
\]
 
Symmetry of squares: squaring numbers modulo $n$ often produces repeated results due to symmetry in the group of residues. This means that different numbers can have the same square when considered modulo $n$. For each $x$, its symmetric counterpart $n-x$ produces the same square modulo $n$. The total number of unique quadratic residues is approximately half of $n$ (or $\floor{n\slash2} + 1$ if 0 is included).

\begin{defn}
    Legendre Symbol: Let \( p \) be a prime number, and let \( a \) be an integer. The \textbf{Legendre symbol} \( \left( \frac{a}{p} \right) \) is defined as follows:

    \[
    \left( \frac{a}{p} \right) =
    \begin{cases}
    1 & \text{if } a \text{ is a quadratic residue modulo } p \text{ and } a \not\equiv 0 \pmod{p}, \\
    -1 & \text{if } a \text{ is a quadratic non-residue modulo } p, \\
    0 & \text{if } p \mid a \text{ (i.e., if } a \equiv 0 \pmod{p}).
    \end{cases}
    \]
\end{defn}

\begin{itemize}
    \item \( \left( \frac{a}{p} \right) = 1 \) means there exists an integer \( x \) such that \( x^2 \equiv a \pmod{p} \) (i.e., \( a \) is a quadratic residue modulo \( p \)). \\
    \item \( \left( \frac{a}{p} \right) = -1 \) means that no such integer \( x \) exists (i.e., \( a \) is a quadratic non-residue modulo \( p \)). \\
    \item \( \left( \frac{a}{p} \right) = 0 \) means that \( a \) is divisible by \( p \) (i.e., \( a \equiv 0 \pmod{p} \)).
    
\end{itemize}

In simpler terms, the Legendre symbol answers the question: \textbf{``Can $a$ be written as the square of some number, when working modulo $p$?"}

To detect squares modulo a prime $p$, we define:
\begin{equation}
    \left( \frac{a}{p} \right) \equiv a^{(p-1)\slash2} \pmod{p}.
\end{equation}

Additional formulae:    
\begin{equation}
    \left( \frac{a}{p} \right) = \left( \frac{a \pmod{p}}{p} \right),
\end{equation}

\begin{equation}
    \left( \frac{a\cdot b}{p} \right) = \left( \frac{a}{p} \right) \left( \frac{b}{p} \right),
\end{equation}

\begin{equation}
    \left( \frac{2}{p} \right) = (-1)^{(p^2 - 1)\slash8},
\end{equation}

\begin{equation}
    \left( \frac{a}{p} \right) = \begin{cases}
        - \left( \frac{p}{a} \right) & \qquad \text{if } p \equiv 3 \pmod{4}, \\
        \left( \frac{p}{a} \right) & \qquad \text{otherwise}.
        \end{cases}
\end{equation}

\textbf{Example:} Compute the Legendre symbol \( \left( \frac{15}{17} \right) \) to check if \( 15 \) is a quadratic residue modulo \( 17 \).

\begin{align*}
    \left( \frac{15}{17} \right) &= \left( \frac{3}{17} \right) \cdot \left( \frac{5}{17} \right) \hfill &\text{by equation (3)} \\
    &= \left( \frac{17}{3} \right) \cdot \left( \frac{17}{5} \right) \hfill &\text{by equation (5)} \\
    &= \left( \frac{2}{3} \right) \cdot \left( \frac{2}{5} \right) \hfill &\text{by equation (2)} \\
    &= (-1)\cdot (-1)^3 \hfill &\text{by equation (4)} \\
    &= 1.
\end{align*}

So $\left( \frac{15}{17} \right) = 1 $, and thus \( 15 \) is a quadratic residue modulo \( 17 \). \\

Instead of manually testing all possible values of \( x \) to see if \( x^2 \equiv a \pmod{p} \), the Legendre symbol gives a quick, direct answer. This is especially helpful when working with large prime numbers.
\begin{itemize}
    \item \textbf{Public-key cryptography} (like RSA) often relies on modular arithmetic and quadratic residues. The Legendre symbol helps in many cryptographic algorithms, such as those involving \textbf{Elliptic Curve Cryptography (ECC)}, \textbf{zero-knowledge proofs}, and \textbf{primality testing}.
    \item For example, in \textbf{the Diffie-Hellman key exchange}, one might need to check if certain numbers are quadratic residues in a modular group.
\end{itemize}

The Legendre symbol above is only defined when its denominator is a prime, but there is a generalization to composite denominators called the Jacobi symbol.

\begin{defn}
    For any integer \( a \) and odd integer \( n \), the \textbf{Jacobi symbol} is defined as the product of the Legendre symbols corresponding to the \emph{prime factors} of $n > 2$: 

    \[
\left( \frac{a}{n} \right) = \left( \frac{a}{p_1} \right)^{e_1} \left( \frac{a}{p_2} \right)^{e_2} \cdots \left( \frac{a}{p_k} \right)^{e_k},
    \]
where
\[ n = p_1^{e_1} \cdot p_2^{e_2} \cdots p_k^{e_k} \]

is the prime factorization of $n$. 
\end{defn}

It is defined as follows:

\[
\left( \frac{a}{n} \right) = 
\begin{cases}
0 & \text{if } n \mid a, \\
1 & \text{if } a \text{ is a quadratic residue modulo } n \text{ and } a \not\equiv 0 \pmod{n}, \\
-1 & \text{if } a \text{ is a quadratic non-residue modulo } n.
\end{cases}
\]

\begin{rem}
If a is square, then the Jacobi symbol will be 1. However, if the Jacobi symbol is 1, a \emph{might} not be a square. 
\end{rem}

\subsection{Primality Tests}
Prime numbers are needed almost always in every public key algorithm. How can you find prime numbers?

\begin{thm}
    \textbf{The Prime Number Theorem:} The number of primes less than X can be given estimated with: 

    \[ \pi(X) \approx \frac{X}{\log(X)} \]
\end{thm}

There are many prime numbers! The probability of a random value to be a prime is $\frac{1}{\log(p)}$. If we need a prime number with $100 \%$ certainty, we need a proof of primality.

\subsubsection{Fermat's Primality Test}
Recall that 
\[ a^{\phi(N)} \equiv 1 \mod N \]

If N is a prime, this equality holds. However, if this equality holds, N is \emph{not necessarily} prime. Probably prime: N is composite with a probability of $\frac{1}{2^k}$. k refers to the number of independent tests or iterations performed to check the primality of a number N. Each test involves choosing a random integer a and checking whether $a^{N-1} \equiv 1 \pmod{N}$.

\begin{rem}
    Carmichael numbers are composite numbers that pass the Fermat primality test for all possible values of a. They are rare but can be problematic in cryptographic applications. They always return \emph{probably prime}. 
\end{rem}

\begin{figure}[h!]
    \centering
    \includegraphics[scale=0.7]{img/fermatprimetest.png}
    \caption{Algorithm for Fermat Primality Test}
\end{figure}

\subsubsection{Miller-Rabin Primality Test}
The Miller-Rabin test is an improvement over the Fermat test. Unlike deterministic primality tests (which can definitively prove whether a number is prime), the Miller-Rabin test provides a result with high probability. If the test declares a number to be composite, then it is definitely not prime. However, if the test declares the number to be prime, there is still a small chance that it is actually composite (this is the ``probabilistic" part). The Miller-Rabin test checks whether a number $n$ passes certain conditions that hold for all prime numbers. It does this by examining the modular arithmetic properties of numbers related to $n$. If $n$ passes these tests, it is likely prime. If it fails, $n$ is definitely composite.

\begin{figure}[h!]
    \centering
    \includegraphics[scale=0.65]{img/MRprimetest.png}
    \caption{Algorithm for Miller-Rabin Test}
\end{figure}

% \begin{enumerate}
%     \item \textbf{Express \( n-1 \) as \( 2^s \cdot d \):}
%     \begin{itemize}
%         \item First, express \( n-1 \) as a product of a power of 2 and an odd integer. Specifically, find \( s \) and \( d \) such that:
%         \[
%         n-1 = 2^s \cdot d, \quad \text{where} \quad d \text{ is odd}.
%         \]
%         \item This can be done by repeatedly dividing \( n-1 \) by 2 until \( d \) becomes odd.
%     \end{itemize}

%     \item \textbf{Choose a random base \( a \):}
%     \begin{itemize}
%         \item Select a random integer \( a \) such that \( 1 < a < n-1 \).
%     \end{itemize}

%     \item \textbf{Compute \( a^d \mod n \):}
%     \begin{itemize}
%         \item Calculate \( a^d \mod n \) using modular exponentiation.
%     \end{itemize}

%     \item \textbf{Check the first condition:}
%     \begin{itemize}
%         \item If \( a^d \equiv 1 \pmod{n} \) or \( a^d \equiv n-1 \pmod{n} \), then \( n \) passes this round of the test.
%     \end{itemize}

%     \item \textbf{Iterate and square:}
%     \begin{itemize}
%         \item If neither of the above conditions hold, square \( a^d \mod n \) repeatedly \( s \) times (i.e., compute \( a^{2^i \cdot d} \mod n \) for \( i = 1, 2, \ldots, s-1 \)).
%         \item Check if any of these intermediate results equals \( n-1 \). If you find \( a^{2^i \cdot d} \equiv n-1 \pmod{n} \) for some \( i \), then \( n \) passes this round.
%     \end{itemize}

%     \item \textbf{Final check:}
%     \begin{itemize}
%         \item If \( n \) does not satisfy the conditions at any stage (i.e., if \( a^{2^i \cdot d} \mod n \neq 1 \) and \( a^{2^i \cdot d} \mod n \neq n-1 \) for all \( i \)), then \( n \) is composite.
%     \end{itemize}

%     \item \textbf{Repeat:}
%     \begin{itemize}
%         \item Repeat steps 2-6 with different random bases \( a \). Each round of testing reduces the probability of error.
%     \end{itemize}
% \end{enumerate}

\subsection{Elliptic Curves}
\begin{defn}
    An elliptic curve is an equation of the form $F: y^2 = x^3 + ax + b \mod p$, with constants a, b.
    \begin{itemize}
        \item $p > 3$, otherwise $x^3 = x$
        \item If P is on F, then also $P + P, P+P+P,$ ... are on F.
    \end{itemize}
\end{defn}

Not all equations make good elliptic curves. They must satisfy a condition that ensures there are no sharp points or self-intersections. The curve must be \emph{smooth} and \emph{non-singular}. This means that the \emph{discriminant must be nonzero}.
\[ 4a^3 + 27b^2 \neq 0 \]

\textbf{Point Addition:} 
\begin{itemize}
    \item Addition of two points $P$ and $Q$ on the curve gives you another point $R$, which is also on the curve. 
            To add \( P = (x_1, y_1) \) and \( Q = (x_2, y_2) \):
            \begin{enumerate}
                \item Draw a straight line through \( P \) and \( Q \). This line will generally intersect the curve at exactly one more point, say \( R' \).
                \item Reflect \( R' \) across the x-axis to get \( R = (x_3, y_3) \), the result of \( P + Q \).
            \end{enumerate}
            The formulas to compute \( R = (x_3, y_3) \) are:
            \[
            x_3 = \lambda^2 - x_1 - x_2
            \]
            \[
            y_3 = \lambda(x_1 - x_3) - y_1
            \]
            where \( \lambda \) (the slope of the line) is:
            \[
            \lambda = \frac{y_2 - y_1}{x_2 - x_1} = \frac{\delta y}{\delta x}
            \]
    \item If you’re adding $P$ to itself (doubling), the line you draw is the tangent to the curve at $P$.
    The formulas for \( R = 2P = (x_3, y_3) \) are:
    \[
    x_3 = \lambda^2 - 2x_1
    \]
    \[
    y_3 = \lambda(x_1 - x_3) - y_1
    \]
    Here, \( \lambda \) (the slope of the tangent) is:
    \[
    \lambda = \frac{3x_1^2 + a}{2y_1}
    \]
    \item If \( P \) and \( Q \) are vertical opposites (e.g., \( P = (x, y) \) and \( Q = (x, -y) \)), the line through them is vertical, and their sum is the \textbf{point at infinity}, \( \mathcal{O} \). Think of \( \mathcal{O} \) as the "zero" for point addition.
    \item Special properties:
    \begin{itemize}
        \item \textbf{Commutative:} \( P + Q = Q + P \)
        \item \textbf{Associative:} \( (P + Q) + R = P + (Q + R) \)
        \item \textbf{Identity Element:} Adding the point at infinity \( \mathcal{O} \) to any point \( P \) gives \( P \) (like adding zero).
    \end{itemize}
\end{itemize}

\begin{figure}[h!]
    \centering    
    \begin{subfigure}[b]{0.3\textwidth}
        \includegraphics[width=\linewidth]{img/ecadd.png}
        \caption{Point Addition}
    \end{subfigure}
    \hfill
    \begin{subfigure}[b]{0.3\textwidth}
        \includegraphics[width=\linewidth]{img/pointdoubling.png}
        \caption{Point Doubling}
    \end{subfigure}
    \hfill
    \begin{subfigure}[b]{0.3\textwidth}
        \includegraphics[width=\linewidth]{img/zeropoint.png}
        \caption{Zero Point}
    \end{subfigure}
\end{figure}

Using elliptic curves lets us create very secure systems with shorter keys (really large prime numbers are not required), which means faster and more lightweight encryption.

\begin{defn}
    \textbf{Elliptic Curve Discrete Log Problem}: For a given integer m and a point P, it is easy to compute $Q = mP$. However, given P and Q, it is hard to compute m.
\end{defn}

Imagine a simple operation: start at one point on the curve, "add" it to itself repeatedly, and you get another point. If I tell you the starting point and the number of additions, it's easy to figure out the end point. But if I give you the end point and ask you to figure out the number of additions, that's really hard. This is what makes elliptic curve cryptography (ECC) secure. For cryptographic validity, the curve must:
\begin{itemize}
    \item Have a sufficiently large number of points (called the order) to ensure security.
    \item Avoid known vulnerabilities (e.g., weak curves susceptible to specific attacks like small subgroup attacks).
\end{itemize}

\newpage

\section{Lecture 3}

\subsection{The Syntax of Encryption}

\subsection{Classical Ciphers}

\subsubsection{Caesar Cipher}

\subsubsection{Shift Cipher}

\subsubsection{Substitution Cipher}

\subsubsection{Vigenère Cipher}

\subsubsection{Permutation Cipher}


\newpage

\section{Lecture 4}

\subsection{Information Theory}
Information theory is the mathematical study of the quantification, storage, and communication of information. By Claude Shannon. 
A key measure in information theory is \emph{entropy}. Entropy quantifies the amount of uncertainty involved in the value of a random variable or the outcome of a random process.
For example, identifying the outcome of a fair coin flip (which has two equally likely outcomes) provides less information (lower entropy, less uncertainty) than identifying the outcome from a roll of a die (which has six equally likely outcomes).


\subsection{Security Definitions}
\begin{defn}
    Computationally Secure: it takes N operations using the \textbf{best known algorithm} to break a crytographic system and N is too large to be feasible.
\end{defn}

\begin{defn}
    Provably Secure: breaking the system is reduced to solving some well-studied hard problem.
\end{defn}

\begin{defn}
    Unconditional Secure/Perfectly Secure: the system is secure against an adversary with unlimited computational power.
\end{defn}

Key size is important. Advances in computer hardware and algorithms are important. In the future, it will be broken due to hardware or better algorithms.

\subsection{Probability and Ciphers}

\begin{defn}
Let $P$ denote the set of plaintexts, $K$ the set of keys, and $C$ denote the set of cipher texts. $p(P=m)$ is the probability that the plaintext is $m$. Then,
\[
p(C=c) = \sum_{k: c\in \mathbb{C}(k)} p(K=k)\cdot p(P=d_k(c))
\]
\end{defn}

This formula calculates the probability of a specific cipher text $C = c$ occuring. It sums over all keys $k$ that could result in the ciphertext $c$ , combining the probabilities of the key and the corresponding plaintext. It is fundamental in several cryptographic analyses:
\begin{itemize}
    \item Ciphertext distribution: compute the distribution of the ciphertexts given the plaintext and key distributions. By ensuring $p(C=c)$ is uniform, cryptographers achieve perfect secrecy in schemes like the One-Time Pad.
    \item Evaluating security metrics: the uncertainty of $C$ is maximized for a secure cipher, ensuring that an attacker cannot infer patterns.
\end{itemize}



\subsection{Perfect Secrecy}
Previously, the ciphertext revelas a lot of information about the plaintext. We want a system in which ciphertext does not reveal anything about the plaintext. 

\begin{defn}
    Perfect secrecy: a cryptosystem has perfect secrecy if \[ p(P=m | C=c) = p(P=m) \] for all plain texts $m$ and ciphertexts $c$.
\end{defn}

\begin{lem}
    Assume the cryptosystem is perfectly secure, then 
    \[ \#K \geq \#C \geq \#P\] where $\#$ denotes the number of items in the corresponding set. 
\end{lem}

\subsubsection{One-Time Pad}

\begin{thm}
    Shannon's Theorem: Let $(P, C, K, e_k(), d_k())$ denote a cryptosystem with \( \#K = \#C = \#P\). Then the cryptosystem provides \emph{perfect secrecy} if and only if:
    \begin{itemize}
        \item Every key is used with equal probability $1/\#K$
        \item For each $m \in P$ and $c \in C$, there is a unique key $k$ such that $c = e_k(m)$
    \end{itemize}
\end{thm}

\begin{defn}
One-Time Pad a cryptographic system that provides perfect secrecy when implemented correctly. It is a symmetric encryption technique (shift cipher) where the sender and receiver use a shared secret key, known as the pad, that is as long as the plaintext message. 
\[\#K = \#P = \#C = 26^n\]
and $p(K=k) = 1/26^n$. Also known as the Vernam cipher. It is perfectly secure because:
\begin{itemize}
    \item The key is truly random
    \item The key is as along as the plaintext
    \item The ciphertext reveals no information about the plaintext. The ciphertext $C$ is independent of the plaintext $P$ without the key $K$, and observing the ciphertext does not reduce the uncertainty about the plaintext.
    \item The key is used only once (hence the name)
\end{itemize}
\end{defn}

\textbf{Example:} Suppose the plaintext \( P \) is "A", and the key \( K \) is a random letter. After encryption, the ciphertext \( C \) could be any letter, with equal probability. If an attacker intercepts \( C = Z \), they cannot determine whether the plaintext \( P \) was "A", "B", or any other letter without the key. Every possible plaintext is equally likely.

\subsection{Entropy}
Due to the key distribution problem (the key must be as long as the message), perfect secrecy is not practical. Instead, we need a cryptosystem in which \textbf{one key can be used many times}, and \textbf{a  small key can encrypt a long message}. Such a system is not perfectly secure, but it should be computationally secure. We need to measure the amount of information first: Shannon's entropy. \\

\begin{defn}
    Shannon's Entropy: Let $X$ be a random variable which takes a finite set of values $x_i$, with $1 \leq 1 \leq n$, and has a probability distribution $p(x)$. We use the convention that if $p_i = 0$ then $p_i \log_2(p_i) = 0$. The entropy of $X$ is defined as:

    \[ H(X) = -\sum_{i=1}^n p_i \cdot \log_2 p_i\] 

    Properties:
    \begin{itemize}
        \item $H(X) \geq 0$
        \item $H(X) = 0$ if $p_i = 1$ and $p_j = 0$ for $i \neq j$
        \item if $p_i = 1/n$ for all $i$, then $H(X) = \log_2(n)$
    \end{itemize}
\end{defn}

\textbf{Example:} For a specific question X: ``Will you go out with me?'', the answer is Yes or No. If you always say No, the amount of information, $H(X) = 0$. If you always say Yes, $H(X) = 0$. You know the result. If you say Yes and No with equal probability, $H(X) = 1$. When you get the answer, no matter what it is, you learn a lot. Here, $H()$ is the entropy, and is independent of the length of $X$.

\subsection{Joint Entropy, Conditional Entropy and Mutual Information}
Used to measure the uncertainty and interdependence between random variables. 

\begin{defn}
    The \textbf{joint entropy} of two random variables $X$ and $Y$ measures the total uncertainty in the pair $(X,Y$), considering them together. The total amount of information contained in \emph{one} observation of both $X$ and $Y$. 
    \[
H(X, Y) = -\sum_{x \in X} \sum_{y \in Y} P(x, y) \log_2 P(x, y)
\]

where $P(x,y)$ is the joint \emph{probability} of $X=x$ and $Y=y$.

\[H(X,Y) \leq H(X) + (HY) \]
\end{defn}

\begin{defn}
The \textbf{conditional entropy} of $X$ given $Y$, $H(X|Y)$, measures uncertainty in $X$ after knowing $Y$. It quantifies how much information about $X$ is still unknown once $Y$ is known.

The entropy of $X$ given an observation of $Y=y$:
\[H(X \mid Y = y) = -\sum_{x} p(X = x \mid Y = y) \cdot \log_2 p(X = x \mid Y = y).
\]

Then,
\[H(M \mid C) = \sum_{c} p(C = c) \cdot H(M \mid C = c) 
= - \sum_{m} \sum_{c} p(C = c) \cdot p(M = m \mid C = c) \cdot \log_2 p(M = m \mid C = c). \]    
\end{defn}

\begin{defn}
    Conditional and joint entropy are connected as follows:
    \[H(X,Y) = H(Y) + H(X|Y) \]
    \[ H(X|Y) \leq H(X)\]
\end{defn}

\begin{defn}
\textbf{Mutual information} quantifies the amount of shared information between $X$ and $Y$. It measures how much knowing $X$ reduces the uncertainty about $Y$ (or vice versa). The expected amount of information that $Y$ gives about $X$ (or $X$ about $Y$).

\[
I(X; Y) = H(X) + H(Y) - H(X, Y) =  H(Y) - H(Y \mid X) = H(X) - H(X \mid Y)
\]

Mutual information represents the reduction in uncertainty about one variable due to knowledge of the other. If $X$ and $Y$ are independent, $I(X; Y)$ because knowing one provides no information about the other.
\end{defn}

\textbf{Example}
Suppose \( X \) and \( Y \) represent the outcomes of rolling two six-sided dice:
\begin{itemize}
    \item \( H(X) \): Entropy of die \( X \) (e.g., how random die \( X \)'s outcomes are).
    \item \( H(X, Y) \): Joint entropy, the uncertainty of the combined outcome of \( X \) and \( Y \).
    \item \( H(Y \mid X) \): Given a roll of \( X \), the remaining uncertainty in \( Y \).
    \item \( I(X; Y) \): Shared information (likely 0 if the dice rolls are independent).
\end{itemize}

\begin{figure}[h!]
    \centering
    \includegraphics[scale=0.4]{img/entropy.png}
\end{figure}

\subsubsection{Application to Ciphers}
\begin{itemize}
    \item $H(P|K,C) = 0$: If you have the cipher text and the key, you can decrypt and obtain the plaintext. There is no surprise, hence 0.
    \item $H(C|P,K) = 0$: When you have the key and the plaintext to can encrypt and obtain the ciphertext. True for deterministic cryptosystems. Not for modern ones that use randomness (the same plaintext can be encrypted with the same key but correspond to a different ciphertext).
    \item $H(K,C) = H(K) + H(P)$
    \item Then, $H(K|C) = H(K,C) - H(C) = H(K)+H(P)-H(C)$. Given the ciphertext, can we obtain information of the key?
\end{itemize}

\subsection{Spurious Keys and Unicity Distance}

\begin{defn}
    Spurious keys: the remaining possible but incorrect keys.
\end{defn}

Tactic of an attacker is to reduce the number of spurious keys to zero. Then the one correct key remains.

Plaintext are not random: natural languages... Random text has an entropy of $H_L = 4.7$ bits (5 bits per letter).
\Comment{continue...}

\subsubsection{Entropy of a Natural Language}
\begin{defn}
    The entropy of a natural language measures the average \emph{amount of information} conveyed per character, word, or sentence in that language, accounting for redundancy and patterns in its structure. It is defined by 
    \[H_L = \lim_{n\rightarrow\infty}\frac{H(P^n)}{n}\]
    where $n$ is the length of letters, and $P^n$ is the n-grams.
\end{defn}

For English (estimation): $1.0 \leq H_L \leq 1.5$ bits per character. This means you need 5 bits of data to represent only 1.5 bits of information. Natural languages are highly redundant (e.g., letter and word patterns), making them less random than completely independent symbols.
The low entropy of natural languages makes cryptographic attacks easier because it introduces predictability and redundancy (useless information) in the plaintext, which attackers can exploit. To mitigate this, cryptographic systems should:
\begin{itemize}
    \item Use strong encryption methods that destroy plaintext patterns (e.g., AES or OTP).
    \item Introduce randomness (e.g., padding).
    \item Avoid predictable plaintext in sensitive communications.
\end{itemize}

\subsubsection{Redundancy}
\begin{defn}
    Redundancy is information that is expressed more than once. The repetition or predictability of information, where certain elements (like letters, words, or phrases) appear more frequently or follow patterns. It is defined as:
    \[ R_L = 1-\frac{H_L}{\log_2 \#\mathbb{P}
    }\]

    where $\#\mathbb{P}$ is the message space.
\end{defn}

\textbf{Example}: For English,
\[R_L \approx 1-\frac{1.25}{\log_2 26} = 0.75 \]
This means 75 percent redundancy, 75 \% of the information in a message is predictable or can be removed without losing meaning. You can zip your files 3 quarters. 

\begin{defn}
    The \textbf{average number of spurious keys} is:
    \[\bar{s}_n \geq \frac{\#\mathbb{K}}{ \#\mathbb{P}^{n\cdot R_L}}-1 \]
    An attacker wants this number to be zero.
\end{defn}

We can make this number zero making the number of keys equal to the number of n-grams times the redundancy. This leads to unicity distance.

\subsubsection{Unicity Distance and Ciphers}
\begin{defn}
    \textbf{Unicity distance} is the average number of cipher texts for which the expected number of spurious keys becomes 0. 
    \[ n_0 \approx \frac{\log_2 \#\mathbb{K}}{R_L \cdot \log_2 \#\mathbb{P}}\]
\end{defn}

In the context of a substition cipher, the unicity distance is the length of the ciphertext required for an attacker to break the cipher with high probability (i.e., to uniquely determine the key). It depends on the number of possible keys and the redundancy of the language used.
Given $\#P=26, \#K=26!, R_L =0.75$, 
\[n_0 \approx \frac{88.4}{0.75 \cdot 4.7} \approx 25\]
To successfully break the cipher (i.e., determine the correct key), an attacker needs to analyze at least 25 characters of ciphertext. This is because, with this many characters, the redundancy and patterns in the language (due to its predictable letter frequencies) will provide enough information for the attacker to determine the key. This highlights the cipher's vulnerability: there are $26!$ possible keys! The smaller the unicity distance, the easier it is to break the cipher with limited ciphertext. \\

For a modern stream cipher $\#P=2$ (binary symbol 0 or 1), $\#K=2^l, R_L =0.75$, then 
\[ n_0 \approx \frac{l}{0.75} = \frac{4\cdot l}{3}\] 

Here, the unicity distance tells us how many ciphertext bits are needed to break the streamcipher. The distance is proportional to the key length $l$. Longer key lengths lead to larger unicity distances, making it harder to break the cipher. \\

With no redundancy, $R_L = 0$. Then:
\[ n_0 \approx \frac{l}{0} = \infty \]

There are no predictable patterns in the plaintext. Perfect randomness would imply that every symbol in the plaintext carries the maximum amount of information, with no repeated or predictable patterns.In such a case, the language would have maximum entropy, and there would be no chance of making guesses about the plaintext from patterns in the ciphertext. In practice: no natural language has zero redundancy.

\newpage

\section{Lecture 5}

\subsection{Security Games}

\subsection{Trapdoor Functions}

\subsection{Notions of Security}
\subsubsection{Basic}
What does it mean for a symmetric encryption scheme to be secure? Adversary should not learn the underlying message: One Way-PASSive attack.

\subsubsection{Modern}
The adversary can also partially break the system. The adversary should not be allowed to obtain \textbf{any} information about the plaintext!

\begin{defn}
Semantic security: like perfect security but the adversary has polynomially bounded computing power. The adversary cannt extract partial knowledge. 
\end{defn}


\begin{defn}
IND Security: If a system is IND secure, it is also semantically secure.
\end{defn}
\subsubsection{Other}
\subsection{Random Oracle Model}

\newpage

\section{Lecture 6}

Randomness is extremely important in cryptography, and the same random values should never be used twice. It is hard to achieve, often we use pseudo-random number generators (PRNGs) to generate randomness. \\

\textbf{Example of Importance of Randomness:}
In a perfect secrecy scheme, the key is as long as the message. If the same key is reused even once, the messages can be decrypted. XORing their ciphertexts effectively removes the key, leaving a direct relationship between the two plaintexts. \\

For a message $m$ and key $k$, the ciphertext is $c = m \oplus k$, where $\oplus$ is the XOR-operation. For two messages encrypted with the same key:

\[c_1 = m_1 \oplus k\] 
\[c_2 = m_2 \oplus k\]

Now, if we XOR the ciphertexts:
\[c_1 \oplus c_2 = (m_1 \oplus k) \oplus (m_2 \oplus k)\]

Using the property of XOR that \(a \oplus a = 0\) and \(a \oplus b \oplus a = b\), we can deduce:

\[
c_1 \oplus c_2 = m_1 \oplus m_2 \oplus k \oplus k
\]

\[
c_1 \oplus c_2 = m_1 \oplus m_2
\]

Thus, the XOR of the two ciphertexts directly reveals the XOR of the two plaintexts. The ``distance" between two messages $m_1$ and $m_2$ represented by their XOR operation $m_1 \oplus m_2$, reveals the bitwise differences between them. This is often inpreted as the Hamming distance (the number of positions at which two strings of the same length differ) between the two messages. The XOR distance not only represents differences but can be exploited to reconstruct messages in insecure systems where the same key is reused (for instance, when the adversary knows one of the plaintexts already and can reconstruct the other).

\subsection{Stream Ciphers}

\begin{defn}
Stream ciphers are based on zeroes and ones. They encrypt bits (bit by bit) instead of blocks of plaintext. Due to bit operations, they are very fast. Easy to implement on hardware and software.
\end{defn}

Consider the following stream cipher:

\[ c = m \oplus F_k(0) \]

Where $F_k(0)$ is a function that generates a keystream based on the key $k$. The keystream is XORed with the plaintext to produce the ciphertext. The keystream is generated by a pseudo-random number generator (PRNG) that takes the key as input. \\

If the key was the same length of the message it would be perfect, but creating such a long key would be very inefficient (only done for government to government communication).
However, we can use a shorter key to mimic this behaviour. The short key is used to choose the output of a PRF (the key is the seed of the PRNG). It will mimic pseudo-randomness long enough (until it repeats itself). The cipher will be passive secure if the PRF is random enough.

\subsection{Linear Feedback Shift Registers (LFSRs)}
We can use LFSRs for generating a binary stream.

\begin{defn}
 A \textbf{Linear Feedback Shift Register} (LFSR) is a sequential shift register that generates a sequence of bits based on a linear feedback function. It consists of:
 \begin{itemize}
    \item A sequence of single-bit registers (0 or 1)
    \item Feedback function: a linear function (usually XOR) that determines how bits are fed back into the register.
    \item Tap positions in the register used in the feedback function.
 \end{itemize}
\end{defn}

\textbf{How it works:}
\begin{enumerate}
    \item Initialization: The register is initialized with a seed value (a non-zero starting state).
    \item Bit Shifting: The register shifts all bits to the right (or left), discarding the last bit.
    \item Feedback Calculation: A new bit is computed based on the XOR of the bits in the ``tapped'' positions.
    \item Repeat: The new bit is inserted into the first position of the register, and the process repeats to generate the next bit in the sequence.
\end{enumerate}

\begin{figure}[h!]
    \centering
    \includegraphics[width=0.5\textwidth]{img/LFSR.png}
    \caption{Feedback shift register}
\end{figure}

The initial state is very important, as it determines the entire sequence (whole sequence of zeroes always results in zeroes). The feedback function and length of the registers are also important.

\subsubsection{Properties of LFSRs}
\begin{itemize}
    \item Periodicity: LFSRs are periodic; they generate a sequence that eventually repeats. The maximum period is $2^n - 1$, where $n$ is the register size, achieved when the feedback taps correspond to a primitive polynomial.
    \item Deterministic: If the initial state (seed) and feedback function are known, the sequence is fully predictable.
    \item Efficiency: LFSRs are computationally efficient, using only shift and XOR operations.
\end{itemize}

\subsubsection{Feedback Functions}
Feedback functions in shift registers are used to compute the new bit that is fed back into the register. The key difference between linear and nonlinear feedback functions lies in how they combine the bits from the register.

\begin{table}[h!]
    \centering
    \renewcommand{\arraystretch}{1.2} % Adjust row spacing
    \begin{tabular}{|>{\raggedright\arraybackslash}m{4cm}|>{\raggedright\arraybackslash}m{5cm}|>{\raggedright\arraybackslash}m{5cm}|}
    \hline
    \textbf{Feature} & \textbf{Linear Feedback} & \textbf{Nonlinear Feedback} \\
    \hline
    \textbf{Operations} & XOR and other linear operations & Nonlinear operations (AND, OR, S-boxes, etc.) \\
    \hline
    \textbf{Predictability} & Easier to predict with known state & Harder to predict, more secure \\
    \hline
    \textbf{Efficiency} & Computationally efficient & More computationally intensive \\
    \hline
    \textbf{Periodicity} & Periodic, with a maximum of $2^n - 1$ & Can be non-periodic or have a very long period \\
    \hline
    \textbf{Applications} & Pseudo-random generators, CRC & Secure encryption, cryptographic systems \\
    \hline
    \end{tabular}
    \caption{Comparison of Linear and Nonlinear Feedback Functions}
    \label{tab:feedback_comparison}
\end{table}

We would like to have a non-linear feedback function, but these are difficult to design. Difficult to estimate whether it is a good non-linear function. When you hit 0 the whole value chain becomes 0, and the cycle is broken. \\

\subsubsection{Zero State in Feedback Functions:}
If an LFSR (or any shift register using feedback functions) reaches the value \textbf{zero}, the system will typically ``lock'' into this state and remain there indefinitely. This is because:

\begin{enumerate}
    \item Feedback Function Output: In a linear feedback system, the XOR of all zeros is zero, meaning that once all bits are zero, the feedback function will continue producing zeros.
    \item State Transition: The state of the LFSR will not change, and the register will stay stuck at zero, generating an output sequence that consists entirely of zeros.
\end{enumerate}

Implications:
\begin{itemize}
    \item \textbf{In LFSRs:}
    \begin{itemize}
        \item The sequence degenerates and loses all pseudo-randomness, becoming a constant stream of zeros.
        \item This reduces the period of the LFSR to just 1, which is undesirable, especially in cryptographic or pseudo-random number generation applications.
    \end{itemize}
    
    \item \textbf{In Cryptographic Systems:}
    \begin{itemize}
        \item If the feedback function hits zero during an encryption or random number generation process, it can render the output insecure.
        \item Attackers can easily exploit the fact that the system produces predictable, non-random output (all zeros).
    \end{itemize}
\end{itemize}

Avoiding the Zero State:
\begin{enumerate}
    \item Seed Initialization: The initial state (seed) of the register is carefully chosen to ensure it is non-zero.
    \item Primitive Polynomials: For LFSRs, the taps (feedback positions) are chosen based on primitive polynomials. This ensures the register cycles through all possible non-zero states ($2^n - 1$) before repeating.
    \item Nonlinear Feedback Functions: Nonlinear systems often include mechanisms to avoid degenerative states like zero, introducing additional complexity to prevent such behavior.
    \item External Checks: Some systems include checks to ensure that a zero state is never reached, restarting the LFSR with a valid non-zero state if necessary.
\end{enumerate}

\subsubsection{Mathematical Expression}
Cell is tapped or not (content of the cell contributes to the feedback function yes or no):
\[
[c_1, c_2, \dots, c_L]
\]

Initial state of the registers:
\[
[s_{L-1}, \dots, s_1, s_0]
\]

Output:
\[
[s_0, s_1, s_2, s_3, \dots, s_{L-1}, s_L, s_{L+1}, \dots]
\]

Then, for \(j \geq L\):
\[
s_j = c_1 \cdot s_{j-1} \oplus c_2 \cdot s_{j-2} \oplus \cdots \oplus c_L \cdot s_{j-L}.
\]

However, we need to see a repetition after a certain period, \(N\), such that:
\[
s_{N+i} = s_i.
\]

\(N\) can be maximum \(2^L - 1\). \\

The \textbf{connection polynomial} $C(X)$ describes how the bits stored in the LFSR are combined to produce the feedback bit, which determines the next state of the LFSR. It is defined as:

\[ C(X) = 1 + c_1 \cdot X + c_2 \cdot X^2 + \dots + c_L \cdot X^L \in \mathbb{F}_2\left[X \right] \]

where $\mathbb{F}_2\left[X \right]$ is the set of polynomials over the binary field (coefficients are either 0 or 1). Proper choice of $C(X)$ can result in a maximal period of $2^L - 1$.\\

The \textbf{characteristic polynomial} $G(X)$ describes the LFSR's output sequence and the polynomial that can be used to generate the entire sequence of output bits. It is a function of the feedback logic defined by the connection polynomial.

\[ G(X) = X^L \cdot C(1/X) \]

\textbf{Example:}

\begin{figure}[h]
    \centering
    \begin{subfigure}{0.45\textwidth}
        \centering
        \includegraphics[width=\textwidth]{img/lfsr1.png}
        \caption{$X^3 + X + 1$}
    \end{subfigure}
    \hfill
    \begin{subfigure}{0.45\textwidth}
        \centering
        \includegraphics[width=\textwidth]{img/lfsr2.png}
        \caption{$X^{32} + X^3 + X + 1$}
    \end{subfigure}
    \caption{Linear Feedback Shift Register Examples}
    \label{fig:lfsr-examples}
\end{figure}

\begin{itemize}
    \item[a)] This is a small LFSR with length $L=3$, described by the connection polynomial $C(X) = X^3 + X + 1$. There are 3 registers: $s_0, s_1, s_2$. The feedback function involves: $X^3$ (feedback from the last register), $X$ (feedback from the first register), and a constant 1. At each step, the contents of the tapped cells ($s_2, s_0$) are XOR-ed to produce a new bit.
    \item[b)] This is a much larger LFSR with length $L=32$, described by the connection polynomial $C(X) = X^{32} + X^3 + X + 1$. There are 32 registers. The feedback function involves the 32nd (last) register, the 3rd register, the 1st register, and a constant 1. The characteristic polynomial $G(X)$ is used to generate the entire output sequence. \Comment{I dont get the exp values for this one??}
\end{itemize}

Note: The connection polynomial determines which registers contribute to the feedback. But is not a direct description of the output stream itself. Instead, it describes the feedback mechanism of the LFSR. \\

How do we know that the feedback function is a good one, such that we achieve the maximum length before the stream cipher repeats itself?

\subsubsection{Primitive Polynomial}
\begin{defn}
A \emph{primitive polynomial} is a polynomial that generates a maximal-length sequence in a finite field, particularly in the context of binary fields $\mathbb{F}_2$ (i.e., polynomials whose coefficients are either 0 or 1). Specifically, for a polynomial to be \textit{primitive}, it must satisfy the following conditions:

\begin{itemize}
    \item The polynomial must be \textbf{irreducible} over $\mathbb{F}_2$ (i.e., it cannot be factored into the product of lower-degree polynomials with coefficients in $\mathbb{F}_2$).
    \item The polynomial must generate a sequence that cycles through all the nonzero elements of the finite field $\mathbb{F}_{2^L}$, where $L$ is the degree of the polynomial.
\end{itemize}
    
In simpler terms, a primitive polynomial produces a sequence of numbers that goes through all possible states except the zero state (when viewed as an LFSR) before repeating.
\end{defn}

When used in LFSRs, primitive polynomials ensure that the generated sequence has a \emph{maximum length} before it repeats. The length of the sequence is $2^L -1$, where $L$ is the degree of the polynomial.

\begin{thm}
    If N is the period, then the characteristic polynomial $f(x)$ is a factor of $1-X^N$. 
\end{thm}


\[ C(X) = 1 + c_1 \cdot X + c_2 \cdot X^2 + \dots + c_L \cdot X^L \in \mathbb{F}_2\left[X \right] \]

Conditions for $C(X)$:
\begin{itemize}
    \item The highest coefficient $c_L$ = 0: the polynomial is singular. This means that $C(X)$ is not a full degree of $L$ and cannot properly generate sequences of maximal length in an LFSR.
    \item The highest coefficient $c_L$ = 1: the polynomial is non-singular. $C(X)$ is a full degree of $L$ and there is a periodicity in the sequences generated by the polynomial. The behaviour of the sequence depends on whether $C(X)$ is irreducible or not.
\end{itemize}

$C(X)$ is irreducible if it cannot be factored into smaller polynomials over $\mathbb{F}_2\left[X \right]$ (other than itself and 1).
Then, the period of the sequence generated by the LFSR is the \emph{smallest} N such that $C(X)$ divides $1 + X^N$.
$N$ will satisfy $N | (2^L-1)$. \\

\textbf{Example:}
3 registers, maximum length is 7:
\[m = 3, p = 2^3 -1 = 7\] 
So we need to find the factors of:
\[1 - x^y = (1-x)(1 + x + x^3)(1 + x^2 + x^3)\]

The divisors of this polynomial (irreducible polynomials) will have a period of 7.

\subsection{Period}

The presence of disjoint cycles in an LFSR, such as in figure \ref{fig:lfsr-period}, has implications for randomness:

\begin{figure}[h]
    \centering
    \begin{subfigure}{0.45\textwidth}
        \centering
        \includegraphics[width=\textwidth]{img/lfsrperiod1.png}
        \caption{$X^4+X^3 + X^2 + 1$}
    \end{subfigure}
    \hfill
    \begin{subfigure}{0.45\textwidth}
        \centering
        \includegraphics[width=\textwidth]{img/lfsrperiod2.png}
        \caption{$X^4 + X + 1$}
    \end{subfigure}
    \caption{State transitions of 4-bit LFSRs with connection polynomials}
    \label{fig:lfsr-period}
\end{figure}

\begin{itemize}
    \item The output sequence splits into two disjoint cycles of length 7 instead of a single cycle of length \( 2^L - 1 \) (15 for \( L = 4 \)).
    \item Reduced Period: The sequence repeats earlier, reducing the overall period and uniformity.
    \item Loss of Randomness: Missing states lead to gaps, making the sequence predictable and non-random.
    \item Cause: The connection polynomial is \textit{not primitive}, preventing the LFSR from achieving a maximum-length sequence.
\end{itemize}

The right polynomial results in a big cycle with all states included, except for the zero state, as expected.

\subsection{Security of LFSRs}
With $L$ registers and $2L$ known output bits, the connection polynomial can be revealed. It is a linear system. If you have sufficient equations, you can solve the linear system.
\begin{itemize}
    \item First $L$ bits reveal the $s$ values
    \item We need to learn $L$ unknowns: $c$ values
\end{itemize}

\[ s_j = \sum^L_{i=1} c_i \cdot s_{j-i} \pmod{2} \]

Therefore, LFSRs are not secure!

\Comment{Linear complexity notes here}

\subsection{Combining LFSRs}
LFSR-based stream ciphers are very fast ciphers, suitable for implementation in hardware, to encrypt real-time data such as voice or video. But they need to be augmented with a method to produce a form of non-linear output.

Create fast and secure non-linear behaviour by combining multiple LFSRs into a non-linear combination function: 

\begin{figure}[h!]
    \centering
    \includegraphics[width=0.5\textwidth]{img/combinelfsr.png}
    \caption{Combining LFSRs}
\end{figure}

\subsubsection{Non-linear Combiners}
Examples covered in lecture are:
\begin{itemize}
    \item Alternating-step generator
    \item Shrinking generator
    \item A5/1
    \item Trivium
    \item RC4
\end{itemize}
\Comment{How much do we need to know about these?}

\newpage

\section{Lecture 7}

\subsection{Block Ciphers}
\begin{defn}
A \textbf{block cipher} is a deterministic cryptographic algorithm that operates on fixed-size blocks of data, typically \( n \) bits, using a symmetric key to perform encryption or decryption. It transforms a plaintext block of size \( n \) bits into a ciphertext block of the same size through a series of reversible, structured operations, and vice versa.


\end{defn}
\begin{figure}[h!]
    \centering
    \includegraphics[width=0.5\textwidth]{img/blockcipher.png}
    \caption{Block Cipher}
    \label{fig:block_cipher}
\end{figure}

where
\begin{itemize}
    \item $m \in \{0,1\}^n$ is the plaintext block,
    \item $k \in K$ is the secret key, chosen from key space K,
    \item $e$ is the encryption function,
    \item $d$ is the decryption function,
    \item $c \in \{0,1\}^b$ is the ciphertext block.
\end{itemize}

\subsubsection{Properties of Block Ciphers}
\begin{enumerate}
    \item Input Data Block: Plaintext is divided into equal-sized blocks.
    \item Encryption/Decryption: Each block undergoes multiple transformations (substitution, permutation, and mixing) to produce ciphertext.
    \item Key: A symmetric key is applied in multiple \emph{rounds} to ensure security.
    \item Padding: If the last block of plaintext is smaller than the block size, padding is added.
    \item Modes of Operation3: Block ciphers work with data larger than one block by using modes like:
    \begin{itemize}
        \item ECB (Electronic Codebook)
        \item CBC (Cipher Block Chaining)
        \item GCM (Galois/Counter Mode)
    \end{itemize}
\end{enumerate}

Block ciphers are typically \emph{64 bits} (in DES), \emph{128 bits} (in AES) or more (in modern ciphers). They should act like a \emph{pseudorandom permutation} (PRP).

\begin{defn}
A \textbf{pseudorandom permutation} (PRP) is a function that defines a bijective (one-to-one and onto) mapping between input and output spaces, such that it is computationally indistinguishable from a truly random permutation when the key is unknown. 
\end{defn}

In order to limit the advantage of the adversary, the key space is kept very large (e.g. $Adv \ 1/|K|$). A block cipher is a \emph{building block} for designing a cipher (a PRP). A block cipher with a \emph{Mode of Operation} is a cipher. The goal is to design an IND-CCA secure cipher.

\subsubsection{Design}
DES and AES are iterated block ciphers. They \emph{repeat a simple round function}. The round $r$ can be fixed or variable. The more rounds, the higher the security of the cipher.
\begin{itemize}
    \item In each round, a round key, derived from the key $k$, is used (by key scheduling algorithm)
    \item The round function should be \emph{invertible}; for decryption the round keys are used in reverse order
    \item In DES: the round is invertible but not the round function
    \item In AES: both the round and the round function are invertible
\end{itemize}

\begin{defn}
    The \textbf{confusion-diffusion} paradigm  is a fundamental design principle for secure cryptographic systems, particularly block ciphers.
    \begin{itemize}
        \item \textbf{Confusion:} Confusion ensures that the relationship between the key and the ciphertext is highly complex, making it difficult for an attacker to infer the key, even if they have access to multiple plaintext-ciphertext pairs. Split the block into smaller blocks and apply a
        substitution on each block.
        \item \textbf{Diffusion:} Diffusion ensures that the influence of a single bit of plaintext (or key) spreads widely over the ciphertext, so that changes in input affect many output bits.  Mix permutations so that local change can effect the
        whole block.
    \end{itemize}
\end{defn} \label{def:confusion_diffusion}

\begin{defn}
    A \textbf{substitution-permutation network} (SPN) is a design model for block ciphers. It consists of a series of linked operations, including substitution, permutation, and key mixing. It is a direct implementation of defintion \ref{def:confusion_diffusion}.
\end{defn}




\subsubsection{The Avalanche Effect}
A small change in the input must affect every bit of the output. This is called the \emph{avalanche effect}. It is a desirable property of block ciphers.
\begin{enumerate}
    \item The S-boxes are designed such that 1 bit effects at least 2 bits in the output of the boxes
    \item The mixing permutations are designed
\end{enumerate}

In principle, you need at least 7 rounds for a good diffusion. \Comment{why?}

\subsection{Feistel Ciphers}

\subsection{Data Encryption Standard (DES)}

\subsubsection{Key Scheduling}

\subsubsection{Security of DES}

\subsection{Advanced Encryption Standard (AES)}

\end{document}